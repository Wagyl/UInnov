%% LyX 2.0.5.1 created this file.  For more info, see http://www.lyx.org/.
%% Do not edit unless you really know what you are doing.
\documentclass[french]{article}
\usepackage[T1]{fontenc}
\usepackage[utf8]{luainputenc}
\usepackage{geometry}
\geometry{verbose}
\usepackage{babel}
\makeatletter
\addto\extrasfrench{%
   \providecommand{\og}{\leavevmode\flqq~}%
   \providecommand{\fg}{\ifdim\lastskip>\z@\unskip\fi~\frqq}%
}

\makeatother
\begin{document}

\title{Projet de Programmation Comparée : ''Interfaces Utilisateurs''}

\maketitle

\section{Qu'est ce qu'une interface utilisateur et comment en programmer}


\subsection{Définition des concepts liés à la notion d'interface utilisateur}


\subsubsection*{Qui sont les utilisateurs ?}
\begin{itemize}
\item Deux catégories : nous utiliserons suivante afin de distinguer les
deux catégories d'utilisateurs : utilisateur application, et programmeur.
\item Dans les deux cas, il faut avoir une approche orientée utilisateur,
les besoins seront différents, mais l'approche à adopter la même. 
\item Utilisateur application : il faut prendre en compte à la fois l'aspect
sensoriel (visuel si écran il y a) et l'aspect ``contrôle'' (gestion
des périphériques d'entrée).
\item Interaction utilisateur application - programme : plusieurs ``niveaux''
de compétences parmi les utilisateurs avec des besoins différents.
\end{itemize}

\subsubsection*{Conception - génie logiciel}

Utilisation de la conception centrée utilisateur qui est un modèle
itératif.
\begin{itemize}
\item \textbf{Analyse :} analyse des besoins, un panel représentatif d'utilisateurs
des deux catégories concernées doit être constitué afin d'établir
les dits besoins.
\item \textbf{Conception :} un prototype doit être conçu en fonction des
besoins établis à l'étape précédente.
\item \textbf{Évaluation :} sur la base du prototype réalisé, une évaluation
est faite. Le procédé étant itératif, cette évaluation servira de
base à la modification des besoins de la première étape. 
\item Il faut établir des ``topic'' d'évaluation - besoins, les deux sont
liés : 

\begin{itemize}
\item Pour l'utilisateur application :

\begin{itemize}
\item Vitesse d'apprentissage - aide nécessaire / intégrée - nombre d'erreurs
commises lors d'un test.
\item Correction des erreurs.
\item Temps de réponse.
\item Efficacité - navigation rapide ?
\end{itemize}
\item Pour le programmeur :

\begin{itemize}
\item Programmation intuitive.
\item Bonne expressivité - ne pas avoir un code verbeux à produire (capacité
d'adaptation au support etc etc ?).
\item Typage fort - sureté.
\item Debugging aisé - analyse de pertinence (détecter le maximum d'absurdités)
\item Intégration aux IDE populaires ? Création d'un IDE ?
\end{itemize}
\end{itemize}
\item \textbf{Les outils : }des outils du génie logiciel peuvent nous aider
à l'intérieur des phases décrites précédemment :

\begin{itemize}
\item Scénarios / diagrammes de cas d'utilisation : à la fois pour établir
les besoins et faire l'évaluation au niveau de l'utilisateur application.
Peut aussi être utilisé pour le côté programmeur bien que le point
suivant me semble être plus approprié.
\item Diagrammes relationnels - diagrammes de classes : relations entre
les différents objets de l'interface, est utile dans la phase de conception
pour ``se mettre à la place'' du programmeur.
\item Design pattern : beaucoup d'interfaces utilisateurs utilisent des
principes similaires d’interaction avec l'utilisateur, l'implémentation
``native'' de comportements génériques peut être pratique.
\end{itemize}
\end{itemize}

\subsection{Division du travail}

La conception d'une interface graphique par dessus un moteur d'application
peut se diviser en quatre parties qui devraient demeurer indépendantes
le plus possible.
\begin{itemize}
\item Communication entre le moteur et l'interface :

\begin{itemize}
\item Affichage : ensemble des données du moteur affichées (sous diverse
forme) par l'interface ;
\item Actions : ensemble des actions de l'utilisateur modifiant l'état du
moteur. Tout élément de l'interface permettant d'agir sur le moteur
devrait être lié à une ''Action''. Une même Action doit pouvoir
être effectuée par différents éléments d'une interface, et par différentes
interfaces.
\end{itemize}

Cette section doit être indépendante de l'interface finale (éléments,
aspect, etc) et de la plateforme.

\item Éléments de l'interface ; dépendant de la plateforme, lié à la partie
précédente.
\item Personnalisation des éléments : positionnement, taille, aspect...


Une telle configuration est évidemment dépendante de la partie précédente.


Certains paramètres de cette configuration doivent pouvoir être modifiés
par l'utilisateur final.


Ces configurations doivent pouvoir être enregistrées et s'échanger
facilement.

\item Aspect général.
\end{itemize}

\section{Analyse de l'existant}

Analyse de Swing / GTK (qui ont évidement des problèmes insurmontables)
ainsi que du couple HTML CSS et le concept intéressant de la séparation
du style de la déclaration des objets.


\section{Outils à développer}


\subsection{Bindings}

Liaison d'une propriété d'un élément de l'interface graphique à une
valeur du moteur de l'application : par exemple, la ''clickabilité''
(oui) d'un bouton à un booléen, le champs d'un label à une chaîne
de caractères, la valeur d'un curseur à une valeur du moteur logique,
le contenu d'un panel à une image...

La liaison doit être effective dans les deux sens : le changement
du champs du label par l'utilisateur doit modifier la valeur de la
chaîne de caractère, tout changement tiers de la valeur doit être
répercuté dans le label de l'interface.\\


Solution : la librairie fournit à l'intention du moteur de l'application
des classes pour des types simples comprenant un pattern observer
à destination des éléments de l'interface. 


\subsection{Actions}

Je prépare juste la sous-section des actions car je vais certainement
en avoir besoin pour le modèle relationnel présenté dessous.


\subsection{Le modèle relationnel}

Les librairies d'interfaces graphiques populaires ne fournissent pas
de moyen haut-niveau de décrire les relations entre les différents
éléments de l'interface utilisateur. Seule une relation de parenté
purement liée au placement des objets les uns dans les autres (le
contenant est automatiquement parent du contenu) est possible, tout
autre système de mise en relation entre les éléments n'est pas directement
supporté. 

Il serait cependant intéressant de pouvoir définir plus finement les
relations entre ces différents éléments, plus précisément, l'idée
serait de ne pas avoir à décrire l'emplacement d'un élément par rapport
à un autre dans le ''code principal'' mais juste les relations entre
ces éléments, un peu à la manière du couple HTML / CSS, une ''feuille
style'' dépendant du support / des préférences de l'utilisateur application
/ du système (cf. langages intermédiaires) se charge d'associer une
relation à un emplacement / visuel / un lien.


\subsection{Langages intermédiaires}

Plus approprié ici, à compléter entre autre quand l'idée d'un interpréteur
pour le visuel de l'interface aura été tranchée.
\end{document}
